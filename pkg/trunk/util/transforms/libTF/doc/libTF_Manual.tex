\documentclass[12pt]{article}

\usepackage{makeidx}
\usepackage{hyperref}
\usepackage{amsmath}
\usepackage{float}
\makeindex

% a todo macro                                                                  
\newcommand{\todo}[1]{\vspace{3 mm}\par \noindent {\textsc{ToDo}}\framebox{
\begin{minipage}[c]{1.0\hsize}\tt #1 \end{minipage}}\vspace{3mm}\par}

\floatstyle{ruled}
\newfloat{program}{thp}{lop}
\floatname{program}{Program}

\newfloat{struct}{thp}{lop}
\floatname{struct}{Data Structure}

\setcounter{tocdepth}{4}


\begin{document}
\title{libTF Manual}
\author{Tully Foote\\
\href{mailto:tfoote@willowgarage.com}{\texttt{tfoote at willowgarage.com}}}
\date{\today}
\maketitle

\tableofcontents
\pagebreak

\section{Overview}
litTF is designed to provide a simple interface for keeping track 
of coordinate transforms within a robotic framework.  
A simple case which was one of the driving design considerations was
the use of a sensor on an actuated platform.  In this case the sensor
will report data in the frame of the moving platform, but the data is much 
more useful in the world frame. To provide the transform libTF will be told 
the position of the platform periodically as it moves.  And when prompted libTF can 
provide the transform from the sensor frame to the world frame at the time when the
sensor data was acquired.  Not only will libTF will keep track of the transform between 
two coordinate frames, but it will also keep track of a whole tree of coordinate frames, 
and automatically chain transformations together between any two connected frames.  

\section{Transform Library}
This section will first provide an overview of terms and conventions followed
by the direct API calls necessary to interact with the library.  

\subsection{Terminology and Conventions}
\paragraph {Coodinate Frame}
\index{Coordinate Frame}
In this documentation the default 
will be a right handed system with X forward, Y left and Z up. 

\paragraph {Denavit-Hartenberg Parameters (DH Parameters)}
\index{Denavit-Hartenberg Parameters} 
DH Parameters are a way to concicely represent a rigid body tranformation in three dimentions.  
It has four parameters: length, twist, offset, and angle.  In addition to using the optimal 
amount of data to store the transformation, there are also optimized methods of chaining 
transformations together.  And lastly the parameters can directly represent rotary and prismatic
joints found on robotic arms.  
See \url{http://en.wikipedia.org/wiki/Denavit-Hartenberg_Parameters} for more details.  

\todo{reproduce labeled diagram a good example \url{http://uwf.edu/ria/robotics/robotdraw/DH_parm.htm}}

\paragraph {Euler Angles}
\index{Euler Angles}
For this library Euler angles are considered to be translations in x, y, z, followed by a rotation around z, y, x.
With the respective angular changes referred to as yaw, pitch and roll. 

\paragraph {Homogeneous Transformation Matrix}
\index{Homogeneous Transform}
Homogeneous Transformation Matrices are a simple way to manipulate 3D translations and rotations 
with standard matrix multiplication.  It is a composite of a standard 3x3 rotation matrix
(see \url{http://en.wikipedia.org/wiki/Rotation_matrix}) and a translation vector.  

Let $_1R_0$ be the 3x3 rotation matrix defined by the Euler angles $(yaw_0,pitch_0,roll_0)$ 
and let $_1T_0$ be the column vector $(x_0,y_0,z_0)^T$ representing the translation.  The combination 
of these two transformations results in the transformation of reference frame 0 to reference frame 1.

A point P in frame 1, $P_1$, can be transformed into frame 1, $P_0$, by the following:

\begin{equation}
\label{eqn:HT}
\begin{pmatrix}
P_0 \\
1
\end{pmatrix}
=
\begin{pmatrix}
_0R_1 & _0T_1 \\
1 & 1
\end{pmatrix}
\begin{pmatrix}
P_1 \\
1
\end{pmatrix}
\end{equation}

The net result is a 4x4 transformation matrix which does both the rotation and translation 
between coordinate frames. The basic approach is to use a 4x1 vector consisting of $(x,y,z,1)^T_1$ 
and by left multiplying by $_0A_1$ it will result in $(x',y',z',1)^T_1$.

The matrix library used within this library is Newmat10.


\paragraph {Newmat10}
\index{Newmat10}
Newmat10 is the matrix library used in this library.  Documentation for Newmat can be found at 
\url{http://www.robertnz.net/nm10.htm}.  

\paragraph {Point}
Within this documentation a point is considered a 3D representation of a position within a 
frame.  It is notated, as x, y, z.  

\paragraph {Time Representation}
libTF uses an \texttt{unsigned long long} (equivilant to a \texttt{uint64\_t}) to represent nanoseconds since the epoch(1970).  
It is named \texttt{ULLtime} for clarity.

rosTF converts automatically to the above representation from a time pair of seconds and nanoseconds since the epoch, to 
be consistant with the rest of ROS.  

\paragraph{Vector}
Within this library a vector is a representation of a direction.  It is represented with the components in 3 directions, 
x,y,z.  However when subject to transformation, a vector will only be subject to rotations and will remain attached to the 
origin.  For example a point (1,1,1) subject to a translation of 5 in the x direction and yawed by 90 degrees would end
at (5,1,1), while a vector (1,1,1) subject to the same transformation would be (-1,1,1).  

\subsection{libTF API}
\label{libTFAPI}
\index{libTF API}
The class which provides external functionality is named TransformReference.  

\subsubsection{Data Types}
\paragraph {Time}
\begin{verbatim}
typedef unsigned long long ULLtime;
\end{verbatim}

\texttt{ULLtime} provides a simple typedef for the timestamp used in this library. 
\texttt{ULLtime} is the number of nanoseconds since the epoch(1970) expressed as 
an unsigned long long, which should be equivilant to a \texttt{uint64\_t}. 

\paragraph{Data} The following data structs are available to facilitate simple 
accessors if the user does not want to use the matrices natively.  

\begin{verbatim}
/** ** Point ****
 *  \brief A simple point class incorperating the time and frameID
 * 
 * This is a point class designed to interact with libTF.  It 
 * incorperates the timestamp and associated frame to make 
 * association easier for the programmer.    
 */
struct TFPoint
{
  double x,y,z;
  unsigned long long time;
  unsigned int frame;
};

/** ** Point2D ****
 *  \brief A simple point class incorperating the time and frameID
 * 
 * This is a point class designed to interact with libTF.  It 
 * incorperates the timestamp and associated frame to make 
 * association easier for the programmer.    
 */
struct TFPoint2D
{
  double x,y;
  unsigned long long time;
  unsigned int frame;
};


/** TFVector
 *  \brief A representation of a vector
 */
struct TFVector
{
  double x,y,z;
  unsigned long long time;
  unsigned int frame;
};

/** TFVector2D
 *  \brief A representation of a 2D vector
 * 
 */
struct TFVector2D
{
  double x,y;
  unsigned long long time;
  unsigned int frame;
};

/** TFEulerYPR
 * \brief A representation of Euler angles
 * Using Yaw, Pitch, Roll
 * commonly known as xyz Euler angles */
struct TFEulerYPR
{
  double yaw, pitch, roll;
  unsigned long long time;
  unsigned int frame;
};

/** TFYaw
 * \brief Rotation about the Z axis.  
 */
struct TFYaw
{
  double yaw;
  unsigned long long time;
  unsigned int frame;
};

/** TFPose
 *  \brief A representation of position in free space
 */
struct TFPose
{
  double x,y,z,yaw,pitch,roll;
  unsigned long long time;
  unsigned int frame;
};

/** TFPose2D
 *  \brief A representation of 2D position
 */
struct TFPose2D
{
  double x,y,yaw;
  unsigned long long time;
  unsigned int frame;
};
\end{verbatim}


\subsubsection{Constructor}
\index{libTF API!Constructor}
\begin{verbatim}
TransformReference(bool interpolating = true, 
                   ULLtime cache_time = DEFAULT_CACHE_TIME,
                   unsigned long long max_extrapolation_distance = DEFAULT_MAX_EXTRAPOLATION_DISTANCE);

\end{verbatim}
This is the constructor for the class.  It's optional argument is 
how long to keep a history of the transforms.  It is expressed in 
\texttt{ULLtime}, nanoseconds since the epoch(1970).  

\subsubsection{Mutators}

\paragraph{setWithEulers}
A method to set the parameters of a coordinate transform with Euler angles. \index{libTF API!setWithEulers}
\begin{verbatim}
void setWithEulers(unsigned int framid, unsigned int parentid, 
                   double x, double y, double z, 
                   double yaw, double pitch, double roll, 
                   unsigned long long time);
\end{verbatim}

\paragraph{setWithDH}
A method to set the parameters of a coordinate transform using DH Parameters. \index{libTF API!setWithDH}
\begin{verbatim} 
void setWithDH(unsigned int framid, unsigned int parentid, 
               double length, double alpha, 
               double offset, double theta, 
               unsigned long long time);
\end{verbatim}



\paragraph{setWithMatrix}
A method to set the parameters of a coordinate transform with a homogeneous transformation matrix. \index{libTF API!setWithMatrix}
\begin{verbatim}
unimplemented
\end{verbatim}

\paragraph{setWithQuaternion}
A method to set the parameters of a coordinate transform with Quaternions. \index{libTF API!setWithQuaternion}
\begin{verbatim}
unimplemented
\end{verbatim}


\subsubsection{Accessors}
\label{libTFAPI accessors}
\paragraph{getMatrix}
\index{libTF API!getMatrix}
\begin{verbatim}
NEWMAT::Matrix getMatrix(unsigned int target_frame,
                         unsigned int source_frame,
                         unsigned long long time); 
\end{verbatim}
getMatrix will return the homogeneous transformation matrix between the souce frame and the target frame.  
The standard approach will return the linearly interpolated 

\paragraph{transform[DATA\_TYPE]}
\index{libTF API!transform[DATA\_TYPE]}
For each of the data types above there are accessors to automatically
transform the data from one frame to another. They are listed below:
\begin{verbatim}
/** Transform a point to a different frame */
TFPoint transformPoint(unsigned int target_frame, 
                       const TFPoint & point_in);

/** Transform a 2D point to a different frame */
TFPoint2D transformPoint2D(unsigned int target_frame, 
                           const TFPoint2D & point_in);

/** Transform a vector to a different frame */
TFVector transformVector(unsigned int target_frame, 
                         const TFVector & vector_in);

/** Transform a 2D vector to a different frame */
TFVector2D transformVector2D(unsigned int target_frame, 
                             const TFVector2D & vector_in);

/** Transform Euler angles between frames */
TFEulerYPR transformEulerYPR(unsigned int target_frame, 
                             const TFEulerYPR & euler_in);

/** Transform Yaw between frames. Useful for 2D navigation */
TFYaw transformYaw(unsigned int target_frame, 
                   const TFYaw & euler_in);

/** Transform a 6DOF pose.  (x, y, z, yaw, pitch, roll). */
TFPose transformPose(unsigned int target_frame, 
                     const TFPose & pose_in);

/** Transform a planar pose, x,y,yaw */
TFPose2D transformPose2D(unsigned int target_frame, 
                         const TFPose2D & pose_in);
\end{verbatim}

\paragraph{viewChain}
\index{libTF API!viewChain}
\begin{verbatim}
std::string viewChain(unsigned int target_frame, 
                      unsigned int source_frame);
\end{verbatim}

\subsubsection{Constants}
\begin{verbatim}
static const unsigned int ROOT_FRAME = 1;  //Hard Value for ROOT_FRAME
static const unsigned int NO_PARENT = 0;  //Value for NO_PARENT

/* The maximum number of frames possible */
static const unsigned int MAX_NUM_FRAMES = 10000;   
/* The maximum number of times to descent before 
 * determining that graph has a loop. */
static const unsigned int MAX_GRAPH_DEPTH = 100; 

//10 seconds in nanoseconds
static const ULLtime DEFAULT_CACHE_TIME = 10 * 1000000000ULL; 

\end{verbatim}

\section{ROS Integration}
The libTF was designed to be used within the ROS architecture and to that end 
a ROS compatible wrapper has also been written.  This wrapper enables the library
to handle all the communication between seperate sources and users of transformations.
The API for this is defined below.

\subsection{libTF ROS API}

\subsubsection {rosTF Client API}
\index{rosTF Client API}
\index{rosTF Client API!Constructor}
\paragraph {Constructor}
The constructor is designed to be launched within a ROS node, but be passed a reference to 
the node so that it can access the nodes public methods.  This is done instead of inheriting
to prevent interactions between different possible future derivatives of ros::node. 
\begin{verbatim}
rosTFClient(ros::node & rosnode);
\end{verbatim}

\index{rosTF Client API!Accessors|see{libTF API}}
\paragraph {Accessors}
rosTF inherits from libTF, so it gathers all data coming and 
and all the accessors remain the same see Section \ref{libTFAPI accessors}.


\subsubsection{rosTF Server}
\index{rosTF Server API}
\index{rosTF Server API!Constructor}
\paragraph {Constructor}
The constructor is designed to be launched within a ROS node, but be passed a reference to 
the node so that it can access the nodes public methods.  This is done instead of inheriting
to prevent interactions between different possible future derivatives of ros::node. 
\begin{verbatim}
rosTFServer(ros::node & rosnode);
\end{verbatim}

\paragraph {Mutators}
The rosTF server class provides three methods to easily broadcast transforms.  

The transform can be defined using Euler angles: (x, y, z, yaw, pitch, roll).
\index{rosTF Server API!sendEuler}
\begin{verbatim}
void sendEuler(unsigned int frame, unsigned int parent, 
               double x, double y, double z, 
               double yaw, double pitch, double roll,
               unsigned int secs, unsigned int nsecs);
\end{verbatim}

The transform can be defined using DH Parameters: (length, twist, offset, angle).
\index{rosTF Server API!sendDH}
\begin{verbatim}
void sendDH(unsigned int frame, unsigned int parent, 
            double length, double twist, 
            double offset, double angle, 
            unsigned int secs, unsigned int nsecs);
\end{verbatim}

Or the transform can be defined using Quaternions: (xt, yt, zt, xr, yr, zr, w).
\index{rosTF Server API!sendQuaternion}
\begin{verbatim}
void sendQuaternion(unsigned int frame, unsigned int parent, 
                    double xt, double yt, double zt, //Translation
                    double xr, double yr, double zr, double w, //Rotation
                    unsigned int secs, unsigned int nsecs);
\end{verbatim}

\section{Example Usage}
\subsection{Library}
%\begin{program}[H]
\begin{verbatim}
#include ``libTF/libTF.h''
#include <sys/time.h>

using namespace std;

int main(void)
{
  double dx,dy,dz,dyaw,dp,dr;
  TransformReference mTR;
  
  //Temporary Variables
  dx = dy= dz = 0;
  dyaw = dp = dr = 0.1;
  
  timeval temp_time_struct;
  gettimeofday(&temp_time_struct,NULL);
  unsigned long long atime = temp_time_struct.tv_sec * 1000000000ULL + (unsigned long long)temp_time_struct.tv_usec * 1000ULL;

  
  //Fill in some transforms
  //  mTR.setWithEulers(10,2,1,1,1,dyaw,dp,dr,atime); //Switching out for DH par
ams below
  mTR.setWithDH(10,2,1.0,1.0,1.0,dyaw,atime);
    //mTR.setWithEulers(2,3,1-1,1,1,dyaw,dp,dr,atime-1000);
   mTR.setWithEulers(2,3,1,1,1,dyaw,dp,dr,atime-100);
   mTR.setWithEulers(2,3,1,1,1,dyaw,dp,dr,atime-50);
   mTR.setWithEulers(2,3,1,1,1,dyaw,dp,dr,atime-1000);
   //mTR.setWithEulers(2,3,1+1,1,1,dyaw,dp,dr,atime+1000);
  mTR.setWithEulers(3,5,dx,dy,dz,dyaw,dp,dr,atime);
  mTR.setWithEulers(5,1,dx,dy,dz,dyaw,dp,dr,atime);
  mTR.setWithEulers(6,5,dx,dy,dz,dyaw,dp,dr,atime);
  mTR.setWithEulers(6,5,dx,dy,dz,dyaw,dp,dr,atime);
  mTR.setWithEulers(7,6,1,1,1,dyaw,dp,dr,atime);
  mTR.setWithDH(8,7,1.0,1.0,1.0,dyaw,atime);
  //mTR.setWithEulers(8,7,1,1,1,dyaw,dp,dr,atime); //Switching out for DH params
 above
  
  
  //Demonstrate InvalidFrame LookupException
  try
    {
      std::cout<< mTR.viewChain(10,9);
    }
  catch (TransformReference::LookupException &ex)
    {
      std::cout << ``Caught `` << ex.what()<<std::endl;
    }
  
  
  // See the list of transforms to get between the frames
  std::cout<<''Viewing (10,8):''<<std::endl;  
  std::cout << mTR.viewChain(10,8);
  
  
  //See the resultant transform
  std::cout <<''Calling getMatrix(10,8)''<<std::endl;
  NEWMAT::Matrix mat = mTR.getMatrix(10,8,atime);  
  std::cout << ``Result of getMatrix(10,8,atime):'' << std::endl << mat<< std::end
l;
  
  //Break the graph, making it loop and demonstrate catching MaxDepthException
  mTR.setWithEulers(6,7,dx,dy,dz,dyaw,dp,dr,atime);
  
  try {
    std::cout<<mTR.viewChain(10,8);
  }
  catch (TransformReference::MaxDepthException &ex)
    {
      std::cout <<''caught loop in graph''<<std::endl;
    }
  
  //Break the graph, making it disconnected, and demonstrate catching Connectivi
tyException
  mTR.setWithEulers(6,0,dx,dy,dz,dyaw,dp,dr,atime);

  try {
    std::cout<<mTR.viewChain(10,8);
  }
  catch (TransformReference::ConnectivityException &ex)
    {
      std::cout <<''caught unconnected frame''<<std::endl;
    }  
  return 0;
};

\end{verbatim}
%\end{program}

\subsection{ROS Implementation}

\subsubsection{Example Broadcaster}
\begin{verbatim}
#include "rosTF/rosTF.h"

class testServer : public ros::node
{
public:
  //constructor
  testServer() : ros::node("server"),count(2){
    pTFServer = new rosTFServer(*this);
  };

  //A pointer to the rosTFServer class
  rosTFServer * pTFServer;


  // A function to call to send data periodically
  void test () {
    pTFServer->sendEuler(5,count++,1,1,1,1,1,1,100000,100000);
    pTFServer->sendDH(5,count++,1,1,1,1,100000,100000);
    pTFServer->sendQuaternion(5,count++,1,1,1,1,1,1,1,100000,100000);
  };

private:
  int count;

};

int main(int argc, char ** argv)
{
  //Initialize ROS
  ros::init(argc, argv);

  //Construct/initialize the server
  testServer myTestServer;
  
  while(myTestServer.ok())
    {
      //Send some data
      myTestServer.test();
      sleep(1);
    }

  return 0;
};
\end{verbatim}

\subsubsection{Example Client}
\begin{verbatim}
#include "rosTF/rosTF.h"

class testListener : public ros::node
{
public:
  //constructor with name
  testListener() : ros::node("client") {
    pClient = new rosTFClient(*this);
  };

  //A pointer to the client library object  
  rosTFClient * pClient;

};


int main(int argc, char ** argv)
{
  //Initialize ROS
  ros::init(argc, argv);

  //Instantiate a local listener
  testListener testListener;
  
  //Nothing needs to be done except wait for a quit
  //The callbacks withing the listener class 
  //will take care of everything
  while(testListener.ok())
    {
      sleep(1);
    }

  return 0;
};
\end{verbatim}

\section{Summary of Internal Mathematics}
This libarary uses quaternion notation as the internal representation
of coordinate transforms.  Transform information is stored in a linked 
list sorted by time.  When a transform is requested the closest two points 
on the linked list are found and then interpolated in time to generate the 
return value.  

\subsection{Storage}
The internal storage for the transforms consists of:
\begin{struct}[H]
\caption{libTF Internal Data Storage}
\begin{verbatim}
  double xtranslation;// The three components of translation
  double ytranslation;
  double ztranslation;
  double xrotation; // The three components of the rotation axis
  double yrotation;
  double zrotation;
  double w; // Omega
  unsigned long long time; //nano seconds since 1970
\end{verbatim}
\end{struct}


\subsection{Interpolation}
The interpolation method used in this library is Spherical Linear Interpolation
(SLERP). \index{SLERP} The standard formula for SLERP is defined in Equation \ref{eq:slerp}.
The inputs are points $p_0$ and $p_1$, and $t$ is the proportion to interpolate 
between $p_0$ and $p_1$, and $\Omega$ is the angle between the axis of the two quaternions. 

\begin{equation}
Slerp(p_0,p_1;t) = \frac{sin((1-t)\Omega)}{sin(\Omega)} * p_0 + \frac{sin(t*\Omega)}{sin(\Omega)} * p_1
\label{eq:slerp}
\end{equation}

\todo{add graphic}

\paragraph{Alternatives}
The SLERP technique was developed in 1985 by Ken Shoemake. \cite{SHOEMAKE} SLERP has been 
largely adopted in the field of computer graphics, however there have been many alternatives 
developed.  The advantages of SLERP are that it provides a constant speed solution 
along the shortest path over the 4D unit sphere.  This is not optimal in all cases however.
A good discussion of various alternatives is here \url{http://number-none.com/product/Understanding%20Slerp,%20Then%20Not%20Using%20It/index.html}. 
The approaches discussed above are normalized linear interpolation (nlerp), and 
log-quaternion linear interpolation (log-quaternion lerp). 
SLERP can be recursively called to do a cubic interpolation, called SQUAD for short.  
This is discussed at \url{http://www.sjbrown.co.uk/?article=quaternions}.


\subsection{Data Representation Conversion}
\subsubsection{Matrix to Quaternion}
\subsubsection{Quaternion to Matrix}
\subsubsection{DH Parameters to Matrix}
\subsubsection{Euler Angles to Matrix}


\bibliographystyle{plain}
\bibliography{libTF_Manual}

\printindex
\end{document}

